\documentclass[a4paper, 12pt]{article}
%Paragraph jumps and indentation
\setlength{\parskip}{1.6em}
\setlength{\parindent}{1.25cm}
% indent block
\newenvironment{my_indent}
 {\xdef\mypind{\the\parindent}\par\begingroup\leftskip\parindent\setlength{\parindent}{0pt}}
 {\par\endgroup\setlength{\parindent}{\mypind{}}}
\usepackage[left=1in, right=1in, top=1in, bottom=1in]{geometry}
%Double spacing
\usepackage{setspace}
\usepackage{indentfirst}
\doublespacing
%Packages
\usepackage{amsmath}
\usepackage{amsthm}
\usepackage[dvipsnames]{xcolor}
\usepackage{mathtools}
\usepackage{amsfonts}
\usepackage{titlesec}
%Images
\usepackage{graphicx}
\graphicspath{ {./images/} }
\usepackage{wrapfig}
\usepackage{float}
%Tables
\usepackage{multirow}
\usepackage{array}
\titleformat{\section}
{\normalfont\large\bfseries}{\thesection}{1em}{}
\titleformat{\subsection}
{\normalfont\large\bfseries}{\thesubsection}{1em}{}
%Equation numbering
\counterwithin{equation}{section}
\usepackage{hyperref}
\urlstyle{same}
\hypersetup{pdfborder=0 0 0}
\usepackage[font=small,labelfont=bf]{caption}
\usepackage{csquotes}
\usepackage[toc,page]{appendix}
\usepackage{svg}
% correct figure and caption
\newcommand{\mycorf}{
    \leftskip-\mypind
    \captionsetup{margin={-\mypind,\mypind}}}

\newtheorem{theorem}{Theorem}
\newtheorem{lemma}[theorem]{Lemma}

\begin{document}

\begin{titlepage}
\begin{center}
\vspace*{4cm}
{\huge\textbf{???}}
\end{center}
\vspace{1cm}
\begin{flushright}
Title\\
Mathematics, Exam Session of 2025\\
\end{flushright}
\end{titlepage}

\tableofcontents
\clearpage

\section{Introduction}
Last year, in January, I participated in the  first round of Netherland Mathematics Olympiad. While the result is satisfactory, there is a problem that no one but me got wrong. So I would really like to prove that I'm better than that, by making the problem harder, and then solve it.
The problem is:

\begin{displayquote}
Xander draws five points and a number of infinitely long lines on an infinite sheet of paper. He does this in such a way that on each line there are at least two of those points and that the lines intersect only at points that Xander has drawn.

What is the maximum number of lines Xander could have drawn?\footnote{\href{https://wiskundeolympiade.nl/phocadownload/opgaven/1e-ronde/2024/Problems.pdf}{https://wiskundeolympiade.nl/phocadownload/opgaven/1e-ronde/2024/Problems.pdf}}
\end{displayquote}

\begin{wrapfigure}{r}{0.25\textwidth}
    \vspace{-0.8\baselineskip}
    \includesvg[width=0.25\columnwidth]{ans1.svg}
    \caption{answer to the original problem}
    \label{fig:ans}
    \vspace{-1\baselineskip}
\end{wrapfigure}
The answer for this question is $6$, as shown in Figure \ref{fig:ans}. The organization also gave a proof\footnote{Which can be found at \href{https://wiskundeolympiade.nl/phocadownload/opgaven/1e-ronde/2024/Solutions.pdf}{https://wiskundeolympiade.nl/phocadownload/opgaven/1e-ronde/2024/Solutions.pdf}. Later in this essay, I will provide a different proof.}. But to prove my ability, I need to make it harder. The question asks for the maximal number of lines given $5$ points. So how about trying to solve the problem for any number of points?

For notation, I will use $n$ to denote the number of points, $m_n$ to denote the maximum number of lines with $n$ points, and I will use ``point(s)" and ``dot(s)" interchangeably. ``graph" will refer to a way to draw the lines and dots.
\clearpage

\section{Solving the problem}
\begin{wrapfigure}{r}{0.25\textwidth}
    \includesvg[width=0.25\columnwidth]{gen_6.svg}
    \caption{general solution with $n=6$}
    \label{fig:gen_6}
\end{wrapfigure}
One method I found efficient in solving those type of problems is trying to prove some random hypothesis. If the hypothesis is true, you are just doing what need to be done. If it's false, it's much easier to find an exception to disprove the hypothesis after trying to prove it and struggle on the weak point of the logic. But to do that, one first need to have an idea what the answer will be like. So I tried to solve the problem for a few given $n$. Soon, I managed to find a solution that seems to work for all $n\geq 3$, as shown in Figure \ref{fig:gen_6}.
% connect the 2 paragrah

By arranging $n-1$ dots in a line and the last one above it, we can draw $n-1$ lines between the last dot and the dots on the line. Together with the line by the $n-1$ dots, we can get $n$ lines in total. Hence, we can establish for $n\geq 3$, that we can draw at least $n$ lines.

And now, I hypothesize that this is also the upper bound. As shown above, this does not apply to $n=5$, where it's possible to draw 6 lines. But I can try to prove it's true for $n>5$.

After some works, I managed to acquire a proof for that. Since my process is a bit messy, I have rearranged the order in which lemmas are presented. For brevity, some less important and obvious conclusion will be explained in the footnote.

\begin{theorem}
\label{theorem:three_intersection}
Given $m>0$ lines lying on a plane, if every line intersects at more than one point, then there exist an intersection of only two lines.
\end{theorem}
% First, I will show that there is no way to construct the graph where each dot is on at least three lines and each line has at least two dots.

\noindent$Proof.$
\vspace{\dimexpr-\parskip-\baselineskip}
\begin{my_indent}
Given $m>0$ lines on a plane. We will choose one line as the base line($l_0$), and construct a coordinate system with y axis parallel to $l_0$. Suppose there is no intersection of less than two lines. 

Consider the intersection that is closest to $l_0$ but does not lies on it (if there are multiple, choose the one with the largest $y$ value). We will name this point A. We can, without loss of generality, assume that point A is to the right of $l_0$. According to our assumption, at least three lines pass through point A.

Such a point must exists. To show so, suppose no such point exists, that is, every intersection lies on $l_0$. Since every line must intersect at at least two place, there must be a line which intersect with $l_0$. However, since two lines can only intersect once, and every intersection in on $l_0$, the new line would intersect only once, creating an contradiction. Hence, the assumption must be false and such a point exists.

\begin{wrapfigure}{r}{0.25\textwidth}
    \mycorf
    \includesvg[inkscapelatex=false,width=0.25\columnwidth]{triple_intersect_para_1.svg}
    \caption{Contradicting points as a result of the assumption}
    \label{fig:triple_intersect_para_1}
    % \vspace{-1\baselineskip}
\end{wrapfigure}
If one of the line that pass through point A is parallel to $l_0$, which we will refer to as $l_1$. Then since we chose A to be the point with the largest $y$ value when distance to $l_0$ is the same, $l_1$ does not have any intersection above point A. At least two other lines pass through point A. Among those lines, we will denote the line with the lowest gradient(most negative) as line $l_2$, the line with the second lowest gradient as $l_3$, and the intersection of $l_0$ and $l_2$ as point B. There must be a third line that pass through point B. However, if the third line has a larger gradient, it would intersects with line $l_1$ at point C, above point A. And if the third line has a lowest gradient than line $l_2$, it would intersect with $l_3$ at point C', which is closer to line $l_0$ than point A, creating a contradiction.

\begin{wrapfigure}{r}{0.25\textwidth}
    \mycorf
    \includesvg[inkscapelatex=false,width=0.25\columnwidth]{triple_intersect_npara_1.svg}
    \caption{Contradicting points as a result of the assumption}
    \label{fig:triple_intersect_npara_1}
    % \vspace{-1\baselineskip}
\end{wrapfigure}
Otherwise, all lines that pass through point A is not parallel to line $l_0$ and must intersect with it. We will call the line with the lowest gradient $l_1$, the line with the second lowest gradient $l_2$ and the line with the largest gradient $l_3$. $l_2$ intersect with $l_0$ at point B. There must be another line that pass through point B. However, such a line must intersect with $l_1$ or $l_2$ in the segment between $l_0$ and point A, creating an intersection closer to line $l_0$, as shown in Figure \ref{fig:triple_intersect_npara_1}. This contradict the assumption that point A is the closest point to $l_0$.

In both cases, we end up with a paradox. Thus, the assumption must be false and an intersection of less than three lines must exist.\qed
\end{my_indent}

Now, we propose that $m$ does not increase faster than $n$, or formally,
\begin{lemma}
    For $n\geq5$, $m_n\leq m_{n-1}+1$.
\end{lemma}

\noindent$Proof.$
\vspace{\dimexpr-\parskip-\baselineskip}
\begin{my_indent}
According to Theorem \ref{theorem:three_intersection}, given a graph with $n\geq1$ and $m_n\geq1$\footnote{Note that since $n\geq5$, $m_n\geq6$ and thus the condition will always be met}, there exists a dot that is on no more than two lines. We will show that in every cases, it is possible to remove the dot and construct a new graph with more than $m_{n}-1$ lines:

Case 1, the dot is on $0$ lines. In this case, we can simply remove the dot. The new graph will have $n-1$ dots and $m_n$ lines. Therefore, $m_{n-1}\geq m_{n}$, or $m_{n}\leq m_{n-1}$.

Case 2, the dot is on $1$ line. In this case, we can remove both the dot and the line. In this case, $m_{n-1}\geq m_{n}-1$, or $m_{n}\leq m_{n-1}+1$.

Case 3, the dot is on $2$ lines, and one of the lines has more than 2 dots lying on it. In this case we can remove the dot and the other line. In this case, $m_{n-1}\geq m_{n}-1$, or, $m_{n}\leq m_{n-1}+1$.\footnote{The reason this proof can't be used for situation in Figure \ref{fig:exception_1} is that it would leave the line with only one dot on it.}

\begin{wrapfigure}{r}{0.25\textwidth}
    \mycorf
    \vspace{-0.5\baselineskip}
    \includesvg[width=0.25\columnwidth]{n4m5_case_1.svg}
    \caption{3 potential lines when both line have only 2 dots}
    \label{fig:exception_1}
    \vspace{-1\baselineskip}
\end{wrapfigure}
 Case 4, the dot is on $2$ lines, and both lines have only 2 dots on them. In this case, there can be only up to 5 lines, as shown in Figure \ref{fig:exception_1}. To prove this is so, consider the vertical line on the left. If there is another line not parallel to it, then they must intersect at the dot at top left. The same reasoning apply to the bottom line, leaving only 3 ways to add a line. Thus, in this case, $m_{n}\leq{5}$.

In the first three case, Thus, $m_{n}\leq\max(m_{n-1}+1,5)$. For $m_{n-1}\geq5$, $m_n\leq m_{n-1}+1$.
Since we already know $m_n\geq n$, for $n\geq5$, $m_{n-1}+1\geq n\ge5$, so $m_n\leq m_{n-1}+1$.
\qed
\end{my_indent}

For $n=3$, it's trivial to see the maximum number of lines is $3$. From the above proof, $m_4\leq\max(m_3+1,5)=5$ and $m_5\leq m_4+1\leq 5+1=6$.

For $n=6$, it can be proven that $m_n=6$. However, since the proof is long and tedious, it can be found in appendix \ref{appendix:proof_n6}.

% And then, we shall show that for $n=6$, $m_n=6$.

% Since $m_5=6$, according to the previous proof, $m_6\leq7$. Suppose $m_6>6$, then $m_6=7$.

% From our first conclusion, we know there exist a dot that lies on 2 or less lines. Since $m_6=m_5+1$, from the second conclusion we know the dot can't lie on 0 lines.

% If the dot lies on 1 line, the removal of the dot would create a graph with $n=5$ and $m=7$. Hence, the removal must have caused the line to have only one dot. So all other lines must either be parallel to the line, or intersect with it at that dot.

% Let the number of lines parallel to it be $a$, with $0\leq a\leq6$. Since $n\leq6$, we get:
% \begin{align}
% n=2+(m-1-a)a&\leq6\\
% -a^2+6a+2&\leq6\\
% (a-3)^2&\geq5\\
% a=0\text{ or }a=6
% \end{align}

% \begin{wrapfigure}{r}{0.25\textwidth}
%     \vspace{-0.5\baselineskip}
%     \includesvg[width=0.25\columnwidth]{n6m7_1_1.svg}
%     \caption{configuration when $a=0$}
%     \label{fig:n6m7_1_1}
%     \vspace{-1\baselineskip}
% \end{wrapfigure}
% If $a=0$ or $a=6$, then some lines would have only one intersection on it, requiring another dot on it. This means $n>6$ as shown in Figure \ref{fig:n6m7_1_1}, contradicting the assumption $n=6$.

% For the situation where the dot lies on two line, instead of proving such a graph does not exist, I decided to use computer to brute force all possibility, and show that such arrangement is impossible. The code can be found in Appendix \ref{appendix:code}.
Since for $n=6$, $m_n=6=n$, by induction, for any $n\geq6$, $n\leq m_n\leq n$. Thus, $m_n=n$.

Hence, the answer to our question is: 

\begin{equation}
m_n= 
\begin{cases}
    0,& \text{if } n=1\\
    1,& \text{if } n=2\\
    3,& \text{if } n=3\\
    5,& \text{if } n=4\\
    6,& \text{if } n=5\\
    n,& \text{otherwise}
\end{cases}
\end{equation}

\clearpage
\section{conclusion}
There you have it. We have not only found the answer to the problem, but also answer for any value of $n$ if they change it. Although it seems they could have only chosen $n=4$ or $n=5$, since other values are all too easy to find. Not too easy to prove, though. Many times I thought I have gotten a good proof, but then found that I didn't consider how multiple lines could intersect at the same point. This drove me to put an upper bound for intersection of lines, leading me to prove that there is no graph such that each line has at least two points with each points on at least three lines. While it's a really interesting conclusion, I feel like the proof could be less messy. 
% For $n=6$, I would prefer having an elegant proof, but most method I thought of have way too many special cases that need to be listed, so I have to resort to writing computer program to assist me. 
Overall, I believe this proof is rather rigorous, although there are certain points that is difficult to convey without drawing a dozen diagram.

% \clearpage
% \section{Work cited}
% \begin{description}
% \item 
% \end{description}

\clearpage
\begin{appendices}

\section{Proof for n=6}
\label{appendix:proof_n6}
stuff


\end{appendices}

\end{document}